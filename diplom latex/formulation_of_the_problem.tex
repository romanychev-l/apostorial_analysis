\specialsection{Постановка задачи}

Для вычисления доверительной ресурсоемкости необходимы данные, полученные многократным запуском программных реализаций исследуемых алгоритмов. После проведения большого числа экспериментов строится доверительный интервал оцениваемой величины ресурсоемкости с заданной доверительной вероятностью.

Данная работа включает в себя следующие этапы:
\begin{enumerate}
\item Выбор подходящего для демонстрации алгоритма.
\item Реализация этого алгоритма и многократный запуск для получения данных.
\item Этап предварительного исследования. Этап необходим для проверки нулевой гипотезы о законе распределения данных.
\item Этап основного исследования.
\item Построение автоматизированной системы для проведения вышеописанного анализа.
\end{enumerate}


\pagebreak