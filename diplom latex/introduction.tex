\specialsection{Введение}

Алгоритмы являются одной из важнейших составляющих эффективной программной системы. В больших проектах правильно написанный алгоритм может сэкономить компании миллионы долларов. Например, в 2016 году социальная сеть ``ВКонтакте'', оптимизировав работу с личными сообщениями, сэкономила минимум 5 млн. долларов [1].

В разработке таких алгоритмов не обойтись без предварительного анализа: нужно уметь оценивать те или иные ресурсы, которые алгоритм потребляет во время работы. Самый распространенный ресурс --- это время выполнения алгоритма. Он является важнейшим, так как пользователи не любят ждать. Второй по значимости ресурс --- это память. Причем, если время выполнения алгоритма вычисляется как количество базовых операций и связано с функцией трудоемкости алгоритма, то память обычно берется не как суммарное количество памяти, которое потребляет алгоритм, а как количество дополнительной памяти, при этом память связана с функцией объема памяти алгоритма.

Обычный асимптотический анализ не всегда точен для конечного диапазона длин входов из-за часто больших коэффициентов у компонентов функций ресурсной эффективности:
$$\Psi_A(D) = C_V \cdot V_A(D) + C_f \cdot f_A(D),$$
где $V$ --- объем потребляемой алгоритмом $A$ памяти, $f$ --- трудоемкость.

В данной работе будет рассмотрен новый подход на основе эмпирического анализа: по данным, полученным экспериментальным путем, будет построена функция доверительной ресурсоемкости (доверительной памяти) с выбранным коэффициентом доверия, а также для упрощения процесса разработана автоматизированная система в виде сайта.

\pagebreak