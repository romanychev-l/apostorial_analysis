\specialsection{Выводы}

Доверительная ресурсоемкость получена для значения коэффициента доверия $\gamma = 0.95$, т. е. в $95\%$ случаев наблюдаемая в единичном эксперименте ресурсоемкость алгоритма не будет превышать значение доверительной ресурсоемкости. Для рассматриваемого алгоритма эти значения меньше ресурсоемкости в худшем случае на исследуемом промежутке длин входа. Это означает, что с помощью апостериорного анализа можно лучше прогнозировать затраты памяти и более эффективно  выбирать оптимальный алгоритм, проводя сравнительный анализ функций доверительной ресурсоемкости.

Созданная автоматизированная система принимает на вход данные работы любого алгоритма и на них проводит анализ. После чего пользователю выводятся результаты анализа.

\pagebreak