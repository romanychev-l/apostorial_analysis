\begin{thebibliography}{1}

\bibitem{voс1} Переписать базу сообщений с нуля и выжить [Электронный ресурс] // URL: url: https://vk.com/blog/messages-database (дата обращения: 20.01.2021)
\bibitem{voс2} Трахтенброт Б. А. Сложность алгоритмов и вычислений. Новосибирск: Изд-во Новосибирского ун-та, 1967.
\bibitem{voс3} Офман Ю. П. Об алгоритмической сложности дискретных функций // ДАН СССР. 1962. Т. 45, вып. 1. С. 48–51.
\bibitem{voс4} Маркова Н. А. Качество программы и его измерения // Системы и средства информатики. Вып. 12. — М.: Наука, 2002. С. 170–188.
\bibitem{voс5} Cormen T. H., Leiserson C. E., Rivest R. L., Stein C. Introduction to Algorithms. Chapter 1: Foundations (Second ed.) // Cambridge, MA: MIT Press and McGraw-Hill. 2001. P. 3–122.
\bibitem{voс6} Кнут Д. Э. Искусство программирования, том 1. Основные алгоритмы, 3-е изд.: Пер. с англ. — М.: Издательский дом «Вильямс», 2002. — 720 с.
\bibitem{voс7} Ахо А., Хопкрофт Дж., Ульман Дж. Построение и анализ вычислительных алгоритмов: Пер. с англ.: — М.: Мир, 1979. — 546 с.
\bibitem{voс8} Петрушин В. Н., Ульянов М. В., Кривенцов А. С. Доверительная трудоемкость — новая оценка качества алгоритмов // Информационные технологии и вычислительные системы. 2009. № 2. С. 23–37.
\bibitem{voс9} Петрушин В. Н., Ульянов М. В. Планирование экспериментального исследования трудоемкости алгоритмов на основе бета-распределения // Информационные технологии и вычислительные системы. 2008. № 2. С. 81–91.
\bibitem{voс10} Hutter F., Xu L. Hoos H., Leyton – Brown K. Algorithm runtime prediction: Methods & evaluation // Artificial Intelligence. 2014. Vol. 206. No 1. P. 79–111.
\bibitem{voс11} Erik D. Demaine, Jayson Lynch, Geronimo J. Mirano, Nirvan Tyagi. Energy-Efficient Algorithms. May 30, 2016
\bibitem{voс12} Петрушин В. Н., Ульянов М. В. Информационная чувствительность компьютерных алгоритмов. М.: Физматлит, 2010.
\bibitem{voс13} Ульянов М. В. Система обозначений в анализе ресурсной эффективности вычислительных алгоритмов // Вестн. МГАПИ. Сер. Естеств. и инж. науки. 2004. No1 (1). С. 42–49.
\bibitem{voс14} Pallottino S. Shortest-path methods: Complexity, interrelations and new propositions // Networks. 1984. Vol. 14. P. 257–267.
\bibitem{voс15} Королюк В. С., Портенко Н. И., Скороход А. В., Турбин А. Ф. Справочник по теории вероятностей и математической статистике. М.: Наука, 1985.
\bibitem{voс16} Репозиторий проекта в системе контроля версий GitHub [Электронный ресурс]: URL: https://github.com/romanychev-l/apostorial\_analysis (дата обращения: 31.05.2021)
\end{thebibliography}